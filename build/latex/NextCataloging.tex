%% Generated by Sphinx.
\def\sphinxdocclass{report}
\documentclass[letterpaper,10pt,english]{sphinxmanual}
\ifdefined\pdfpxdimen
   \let\sphinxpxdimen\pdfpxdimen\else\newdimen\sphinxpxdimen
\fi \sphinxpxdimen=.75bp\relax

\PassOptionsToPackage{warn}{textcomp}
\usepackage[utf8]{inputenc}
\ifdefined\DeclareUnicodeCharacter
 \ifdefined\DeclareUnicodeCharacterAsOptional
  \DeclareUnicodeCharacter{"00A0}{\nobreakspace}
  \DeclareUnicodeCharacter{"2500}{\sphinxunichar{2500}}
  \DeclareUnicodeCharacter{"2502}{\sphinxunichar{2502}}
  \DeclareUnicodeCharacter{"2514}{\sphinxunichar{2514}}
  \DeclareUnicodeCharacter{"251C}{\sphinxunichar{251C}}
  \DeclareUnicodeCharacter{"2572}{\textbackslash}
 \else
  \DeclareUnicodeCharacter{00A0}{\nobreakspace}
  \DeclareUnicodeCharacter{2500}{\sphinxunichar{2500}}
  \DeclareUnicodeCharacter{2502}{\sphinxunichar{2502}}
  \DeclareUnicodeCharacter{2514}{\sphinxunichar{2514}}
  \DeclareUnicodeCharacter{251C}{\sphinxunichar{251C}}
  \DeclareUnicodeCharacter{2572}{\textbackslash}
 \fi
\fi
\usepackage{cmap}
\usepackage[T1]{fontenc}
\usepackage{amsmath,amssymb,amstext}
\usepackage{babel}
\usepackage{times}
\usepackage[Bjarne]{fncychap}
\usepackage{sphinx}

\usepackage{geometry}

% Include hyperref last.
\usepackage{hyperref}
% Fix anchor placement for figures with captions.
\usepackage{hypcap}% it must be loaded after hyperref.
% Set up styles of URL: it should be placed after hyperref.
\urlstyle{same}
\addto\captionsenglish{\renewcommand{\contentsname}{Contents:}}

\addto\captionsenglish{\renewcommand{\figurename}{Fig.}}
\addto\captionsenglish{\renewcommand{\tablename}{Table}}
\addto\captionsenglish{\renewcommand{\literalblockname}{Listing}}

\addto\captionsenglish{\renewcommand{\literalblockcontinuedname}{continued from previous page}}
\addto\captionsenglish{\renewcommand{\literalblockcontinuesname}{continues on next page}}

\addto\extrasenglish{\def\pageautorefname{page}}

\setcounter{tocdepth}{1}



\title{Next Cataloging Documentation}
\date{Mar 27, 2019}
\release{}
\author{NEKLS staff}
\newcommand{\sphinxlogo}{\vbox{}}
\renewcommand{\releasename}{}
\makeindex

\begin{document}

\maketitle
\sphinxtableofcontents
\phantomsection\label{\detokenize{index::doc}}



\chapter{Introduction}
\label{\detokenize{README:introduction}}\label{\detokenize{README::doc}}
This is the new repository for Next Kansas cataloging training. It will
replace all previous training documents on this subject.


\chapter{Classification}
\label{\detokenize{classification:classification}}\label{\detokenize{classification::doc}}
Using the NExpress catalog to find items shelved as member libraries
requires that items follow some basic classification rules. The NExpress
classification scheme is based on the item’s home library, shelving
location, item type, collection code, and call number \textendash{} in that order.

A visual overview of how items are classified would look something like
this:

\noindent\sphinxincludegraphics{{010-classification-opac}.jpg}


\section{Home library:}
\label{\detokenize{classification:home-library}}
Each item \sphinxstylestrong{*must*} include a home library that is defined by the
drop-down list of current NExpress libraries. Library codes and names
are set at the system level and cannot be added, changed, or updated by
staff at member libraries.

Patrons will only ever see a library’s name in the OPAC. Staff will
usually see library names but may also see library codes - especially
when running reports.

The current library codes/names are:


\begin{savenotes}\sphinxatlongtablestart\begin{longtable}{|l|l|}
\hline
\sphinxstyletheadfamily 
Library code
&\sphinxstyletheadfamily 
Library name
\\
\hline
\endfirsthead

\multicolumn{2}{c}%
{\makebox[0pt]{\sphinxtablecontinued{\tablename\ \thetable{} -- continued from previous page}}}\\
\hline
\sphinxstyletheadfamily 
Library code
&\sphinxstyletheadfamily 
Library name
\\
\hline
\endhead

\hline
\multicolumn{2}{r}{\makebox[0pt][r]{\sphinxtablecontinued{Continued on next page}}}\\
\endfoot

\endlastfoot

ATCHISON
&
Atchison Public Library
\\
\hline
BALDWIN
&
Baldwin City Public Library
\\
\hline
BASEHOR
&
Basehor Community Library
\\
\hline
BERN
&
Bern Community Library
\\
\hline
BONNERSPGS
&
Bonner Springs City Library
\\
\hline
BURLINGAME
&
Burlingame Community Library
\\
\hline
CARBONDALE
&
Carbondale City Library
\\
\hline
CENTRALIA
&
Centralia Community Library
\\
\hline
CORNING
&
Corning City Library
\\
\hline
DIGITAL
&
Digital Content
\\
\hline
DONIELWD
&
Doniphan County Library - Elwood
\\
\hline
DONIHIGH
&
Doniphan County Library - Highland
\\
\hline
DONITROY
&
Doniphan County Library - Troy
\\
\hline
DONIWATH
&
Doniphan County Library - Wathena
\\
\hline
EFFINGHAM
&
Effingham Community Library
\\
\hline
EUDORA
&
Eudora Public Library
\\
\hline
EVEREST
&
Everest, Barnes Reading Room
\\
\hline
HIAWATHA
&
Hiawatha, Morrill Public Library
\\
\hline
HOLTON
&
Holton, Beck-Bookman Library
\\
\hline
HORTON
&
Horton Public Library
\\
\hline
LANSING
&
Lansing Community Library
\\
\hline
LEAVENWRTH
&
Leavenworth Public Library
\\
\hline
LINWOOD
&
Linwood Community Library
\\
\hline
LOUISBURG
&
Louisburg (Hoopla Digital Only)
\\
\hline
LYNDON
&
Lyndon Carnegie Library
\\
\hline
MCLOUTH
&
McLouth Public Library
\\
\hline
MERIDEN
&
Meriden-Ozawkie Public Library
\\
\hline
NEKLS
&
Northeast Kansas Library System
\\
\hline
NORTONVLLE
&
Nortonville Public Library
\\
\hline
OSAGECITY
&
Osage City Library
\\
\hline
OSAWATOMIE
&
Osawatomie Public Library
\\
\hline
OSKALOOSA
&
Oskaloosa Public Library
\\
\hline
OTTAWA
&
Ottawa Library
\\
\hline
OVERBROOK
&
Overbrook Public Library
\\
\hline
PAOLA
&
Paola Free Library
\\
\hline
PERRY
&
Perry-Lecompton Community Library
\\
\hline
PHAXTELL
&
Prairie Hills Schools - Axtell Public School
\\
\hline
PHSES
&
Prairie Hills Schools - Sabetha Elementary School
\\
\hline
PHSHS
&
Prairie Hills Schools - Sabetha High School
\\
\hline
PHSMS
&
Prairie Hills Schools - Sabetha Middle School
\\
\hline
PHWAC
&
Prairie Hills Schools - Wetmore Academic Center
\\
\hline
POMONA
&
Pomona Community Library
\\
\hline
RICHMOND
&
Richmond Public Library
\\
\hline
ROSSVILLE
&
Rossville Community Library
\\
\hline
SABETHA
&
Sabetha, Mary Cotton Library
\\
\hline
SENECA
&
Seneca Free Library
\\
\hline
SILVERLAKE
&
Silver Lake Library
\\
\hline
TONGANOXIE
&
Tonganoxie Public Library
\\
\hline
VALLEYFALL
&
Valley Falls, Delaware Township Library
\\
\hline
WELLSVILLE
&
Wellsville City Library
\\
\hline
WETMORE
&
Wetmore Public Library
\\
\hline
WILLIAMSBG
&
Williamsburg Community Library
\\
\hline
WINCHESTER
&
Winchester Public Library
\\
\hline
\end{longtable}\sphinxatlongtableend\end{savenotes}


\section{Shelving location:}
\label{\detokenize{classification:shelving-location}}
Each item record \sphinxstyleemphasis{should} include a shelving location that is defined by
the drop-down list of current NExpress shelving locations. Shelving
locations are set at the system level and cannot be added, changed, or
updated by staff at member libraries.

Rather than being names of actual shelving locations, NExpress has
primarily used the shelving location to denote whether an item is an
“Adult,” “Childrens,” or “Young adult” item for the purposes of
gathering statistical data. Virtually every item in NExpress has an
“Adult,” “Childrens,” or “Young adult.”

However, there are a handful of other shelving locations. Some of them
are locations created by the software for system use and some have been
created to help specific libraries gather statistical data on display
items.

Additionally, it is currently possible to add an item with no shelving
location. This is not recommended because it will cause problems with
your library’s monthly and annual reports.

Patrons will only ever see a shelving location’s name in the OPAC. Staff
will usually see shelving location’s name but may also see shelving
location codes - especially when running reports.

The current shelving location codes/names are:


\begin{savenotes}\sphinxattablestart
\centering
\begin{tabulary}{\linewidth}[t]{|T|T|}
\hline
\sphinxstyletheadfamily 
Location code
&\sphinxstyletheadfamily 
Location name
\\
\hline
ADULT
&
Adult
\\
\hline
CART
&
Book Cart
\\
\hline
CATALOGING
&
Cataloging
\\
\hline
CHILDRENS
&
Childrens
\\
\hline
LVPLADULT
&
\textasciitilde{}Adult Display (Leavenworth Public Library)
\\
\hline
LVPLCHILD
&
\textasciitilde{}Children’s Display (Leavenworth Public Library)
\\
\hline
LVPLYA
&
\textasciitilde{}Young Adult Display (Leavenworth Public Library)
\\
\hline
PROC
&
Processing Center
\\
\hline
YOUNGADULT
&
Young Adult
\\
\hline
\end{tabulary}
\par
\sphinxattableend\end{savenotes}


\section{Item type:}
\label{\detokenize{classification:item-type}}
Each item record \sphinxstylestrong{*must*} include an item type that is defined by the
drop-down list of current NExpress item types. Item types are set at the
system level and cannot be added, changed, or updated by staff at member
libraries.

An item record cannot be saved without an item type. Beyond describing
an item, the major function of an item’s type is to control its
circulation rules.

Patrons will only ever see an item type name in the OPAC. Staff will
usually see an item type name but may also see item type codes -
especially when running reports.

The current shelving location codes/names are:


\begin{savenotes}\sphinxattablestart
\centering
\begin{tabulary}{\linewidth}[t]{|T|T|}
\hline
\sphinxstyletheadfamily 
Item type code
&\sphinxstyletheadfamily 
Item type name
\\
\hline
AUDIOBOOK
&
AUDIOBOOK
\\
\hline
BOOK
&
BOOK
\\
\hline
COMPUTER
&
PUBLIC USE COMPUTER
\\
\hline
DIGITAL
&
DIGITAL CONTENT
\\
\hline
EQUIPMENT
&
EQUIPMENT
\\
\hline
FLPLAYAWAY
&
FLOATING PLAYAWAY (SLK)
\\
\hline
GAME
&
VIDEOGAME
\\
\hline
ILL
&
BORROWED FROM ANOTHER LIBRARY
\\
\hline
KITS
&
KIT OR MULTI-PART SET
\\
\hline
LAUNCHPAD
&
LAUNCHPAD TABLET
\\
\hline
LOCALHOLD
&
LOCAL HOLD ONLY
\\
\hline
LOCALHOLD1
&
LOCAL HOLD MOVIE
\\
\hline
LOCALHOLD2
&
LOCAL HOLD BOOK
\\
\hline
MAGAZINE
&
MAGAZINE
\\
\hline
MEDIA
&
MOVIES
\\
\hline
MIFI
&
MIFI DEVICE
\\
\hline
MUSIC
&
MUSIC CD
\\
\hline
NEWAUDIO
&
NEW AUDIOBOOK
\\
\hline
NEWBOOK
&
NEW BOOK
\\
\hline
NEWMAG
&
NEW MAGAZINE
\\
\hline
NEWMEDIA
&
NEW MOVIE
\\
\hline
NPASS
&
PASS
\\
\hline
REFERENCE
&
REFERENCE MATERIAL
\\
\hline
WALKIN
&
WALKIN ONLY
\\
\hline
WALKIN1
&
WALKIN MOVIE
\\
\hline
WALKIN2
&
WALKIN BOOK
\\
\hline
XXX
&
(UNCLASSIFIED)
\\
\hline
\end{tabulary}
\par
\sphinxattableend\end{savenotes}


\section{Collection code:}
\label{\detokenize{classification:collection-code}}
Each item record \sphinxstyleemphasis{should} include a collection code that is defined by
the drop-down list of current NExpress collection codes. Collection
codes are set at the system level and cannot be added, changed, or
updated by staff at member libraries.

Patrons will only ever see collection code names in the OPAC. Staff will
usually see collection code names but may also see collection code codes
- especially when running reports.

The current shelving location codes/names are:


\begin{savenotes}\sphinxatlongtablestart\begin{longtable}{|l|l|}
\hline
\sphinxstyletheadfamily 
Collection code
&\sphinxstyletheadfamily 
Collection code name
\\
\hline
\endfirsthead

\multicolumn{2}{c}%
{\makebox[0pt]{\sphinxtablecontinued{\tablename\ \thetable{} -- continued from previous page}}}\\
\hline
\sphinxstyletheadfamily 
Collection code
&\sphinxstyletheadfamily 
Collection code name
\\
\hline
\endhead

\hline
\multicolumn{2}{r}{\makebox[0pt][r]{\sphinxtablecontinued{Continued on next page}}}\\
\endfoot

\endlastfoot

BAKEWARE
&
Bakeware
\\
\hline
BILINGUAL
&
Bilingual
\\
\hline
BIOGRAPHY
&
Biography
\\
\hline
BLU-RAY
&
Blu-Ray
\\
\hline
BOARDBK
&
Board Book
\\
\hline
BOOKONCASS
&
Book on Cassette
\\
\hline
BOOKONCD
&
Book on CD
\\
\hline
BOOKONMP
&
Book on Digital (Playaways)
\\
\hline
COMBO
&
Blu/Ray - DVD Combo Pack
\\
\hline
COMPUTER
&
Public Computer
\\
\hline
DISPLAY
&
On Display
\\
\hline
DLAUDIO
&
Downloadable Audiobook
\\
\hline
DLBOOK
&
Downloadable Book
\\
\hline
DLGRAPHIC
&
Downloadable Graphic Novel
\\
\hline
DLMAG
&
Downloadable Magazine
\\
\hline
DLMUSIC
&
Downloadable Music
\\
\hline
DLTV
&
Downloadable TV Show
\\
\hline
DLVIDEO
&
Downloadable Movie
\\
\hline
DVD
&
DVD
\\
\hline
EASY
&
Picture Book
\\
\hline
EDUCATION
&
Education
\\
\hline
ERESOURCE
&
Online Resource
\\
\hline
FICTION
&
Fiction
\\
\hline
GADGET
&
Gadget
\\
\hline
GAMEBOY
&
VG-Gameboy
\\
\hline
GAMECUBE
&
VG-Nintendo Gamecube
\\
\hline
GENEALOGY
&
Genealogy
\\
\hline
GRAPHIC
&
Graphic Novel
\\
\hline
HOLIDAY
&
Holiday
\\
\hline
ILL
&
Interlibrary Loan
\\
\hline
INSPRATION
&
Inspirational
\\
\hline
KANSAS
&
Kansas
\\
\hline
LARGEPRINT
&
Large Print
\\
\hline
LARGEPRNF
&
Large Print Non-Fiction
\\
\hline
MAGAZINE
&
Magazine
\\
\hline
MAP
&
Map
\\
\hline
MISC
&
Miscellaneous
\\
\hline
MUSIC
&
Music CD
\\
\hline
MYSTERY
&
Mystery
\\
\hline
NINTENDODS
&
VG-Nintendo DS
\\
\hline
NINTNDO3DS
&
VG-Nintendo 3DS
\\
\hline
NONFICTION
&
Non-Fiction
\\
\hline
OVERSIZE
&
Oversize
\\
\hline
PAPERBACK
&
Paperback
\\
\hline
PARENTING
&
Parenting
\\
\hline
PROFCOLL
&
Professional Collection
\\
\hline
PS2
&
VG-PlayStation 2 (PS2)
\\
\hline
PS3
&
VG-PlayStation 3 (PS3)
\\
\hline
PS4
&
VG-PlayStation 4 (PS4)
\\
\hline
PSP
&
VG-PlayStation Portable (PSP)
\\
\hline
PUZZLESBDG
&
Puzzles \& Board Games
\\
\hline
READER
&
Easy Reader
\\
\hline
ROMANCE
&
Romance
\\
\hline
SF-FANT
&
Sci Fi-Fantasy
\\
\hline
SOFTWARE
&
Software
\\
\hline
SPANISH
&
Spanish
\\
\hline
SWITCH
&
VG-Nintendo Switch
\\
\hline
TVSERIES
&
TV Series
\\
\hline
VHS
&
VHS
\\
\hline
VIDEOGAME
&
VG\textendash{}Videogame
\\
\hline
WESTERN
&
Western
\\
\hline
WII
&
VG-Nintendo Wii
\\
\hline
WIIU
&
VG-Nintendo Wii U
\\
\hline
XBOX
&
VG-XBox
\\
\hline
XBOX360
&
VG-XBox 360
\\
\hline
XBOXONE
&
VG-XBox One
\\
\hline
XXX
&
(Unclassified)
\\
\hline
\end{longtable}\sphinxatlongtableend\end{savenotes}


\section{Item call number:}
\label{\detokenize{classification:item-call-number}}
Each item record \sphinxstyleemphasis{should} include a call number. Call numbers are not
set at the system level and must be added by staff at the library
cataloging an item. The call numbers used should follow the rules each
library uses in defining their call numbers.

If a bibliographic record has a call number in the
080a,
the call number field in the item record will be blank.


\section{Screenshots}
\label{\detokenize{classification:screenshots}}
In the OPAC, the classification scheme looks like this:

\noindent\sphinxincludegraphics{{020-classification-opac}.jpg}

OPAC screenshot - classification
In the staff client, the classification scheme looks like this:

\noindent\sphinxincludegraphics{{030-classification-opac}.jpg}

OPAC screenshot - classification
In the new OPAC, the classification scheme will look like this:

\noindent\sphinxincludegraphics{{040-classification-opac}.jpg}

New OPAC screenshot - classification


\chapter{Searching for a title}
\label{\detokenize{searching-for-a-title:searching-for-a-title}}\label{\detokenize{searching-for-a-title::doc}}

\section{Search by ISBN}
\label{\detokenize{searching-for-a-title:search-by-isbn}}\begin{enumerate}
\item {} 
Select “More \textgreater{} Cataloging”

\end{enumerate}

\noindent\sphinxincludegraphics{{050-searching}.png}

B. Search for the item by ISBN or or UPC number (you can usually do this
simply by scanning the UPC label on the item).

\noindent\sphinxincludegraphics{{060-searching}.png}

C. If you find results, verify that the items actually match by
verifying the ISBN in the
020a but also verify that the
title, author, year, publisher, physical description (page numbers,
large print, etc.) match.

\noindent\sphinxincludegraphics{{070-searching}.png}

\noindent\sphinxincludegraphics{{080-searching}.png}

D. If you have a match, you can just add the item you’re cataloging to
the existing record.

\sphinxhref{title-already-exists/adding-an-item.md}{Click here to see how to add an item to an existing
record}

E. If you do not find a match on the ISBN or UPC number, you should then
do a search by title/author. It is possible that a copy exists that
matches in every way except ISBN or UPC number. This can especially be
true for items that are quite old or have been self-published.

\noindent\sphinxincludegraphics{{090-searching}.png}

F. If you find a match based on title/author, you should also verify
that the year, publisher, physical description (page numbers, large
print, etc.) match.

\noindent\sphinxincludegraphics{{070-searching}.png}

\noindent\sphinxincludegraphics{{080-searching}.png}

G. If you have a match, you can just add the item you’re cataloging to
the existing record.

\sphinxhref{title-already-exists/adding-an-item.md}{Click here to see how to add an item to an existing
record}

H. If you do not find a match by title/author, your next step would be
to try to copy-catalog the title.

\sphinxhref{copy-cataloging/}{Click here to see how to copy catalog}


\chapter{Title already exists}
\label{\detokenize{title-already-exists/README:title-already-exists}}\label{\detokenize{title-already-exists/README::doc}}
If a title already exists, all you have to do is add a new item to the
existing record.


\chapter{Adding an item}
\label{\detokenize{title-already-exists/adding-an-item:adding-an-item}}\label{\detokenize{title-already-exists/adding-an-item::doc}}

\section{Adding a single item}
\label{\detokenize{title-already-exists/adding-an-item:adding-a-single-item}}
A. \sphinxhref{../searching-for-a-title.md}{After searching for a title}, if you
find that a matching title already exists, click on the action button of
that record and select + Add/Edit items.
\begin{enumerate}
\setcounter{enumi}{1}
\item {} 
On the Add/Edit screen, you must update the following fields:

\end{enumerate}
\begin{itemize}
\item {} 
Collection code

\item {} 
Home Library (should default to your library)

\item {} 
Current location (should default to your library)

\item {} 
Shelving location

\item {} 
Call number (will default to data from 082\$a)

\item {} 
Barcode

\item {} 
Cost, replacement price

\item {} 
Koha item type

\end{itemize}
\begin{enumerate}
\setcounter{enumi}{2}
\item {} 
Additionally you can add these optional fields:

\end{enumerate}
\begin{itemize}
\item {} 
Source of acquisition

\item {} 
Cost, normal purchase price

\item {} 
Copy number

\item {} 
Non-public note

\item {} 
Public note

\end{itemize}

D. If you only have one item to add to the record, once the fields are
filled, click on “Add item” to finish adding the item.


\section{Adding 2 items to one records}
\label{\detokenize{title-already-exists/adding-an-item:adding-2-items-to-one-records}}
E. If you have two copies of the same title to add to a record, once
you’ve finished adding the first, click “Add and duplicate” to add a
second copy and follow the instructions below.

The fields on the second item will match the details you updated on the
first item except the barcode number.


\section{Adding more than 2 items to one records}
\label{\detokenize{title-already-exists/adding-an-item:adding-more-than-2-items-to-one-records}}
F. If you have more than 2 copies of the same title to add to a record
and your barcode numbers are in sequence (1, 2, 3, 4 …), once you’ve
finished adding the first, click “Add multiple copies of this item.”

A new field will open that will allow you to enter the total number of
items that you’re going to add to this record.

All of the items you add using this method will have barcode nubers
added in sequence starting with the first number you enter. For example,
if you are adding 5 copies and the first has barcode number 1006, the
items will have barcode numbers 1006, 1007, 1008, 1009, and 1010.


\chapter{Copy cataloging}
\label{\detokenize{copy-cataloging/README:copy-cataloging}}\label{\detokenize{copy-cataloging/README::doc}}
99\% of the cataloging we do in NExpress is copy cataloging.

These next pages explain how to copy catalog from the different sources
of records we get most of our records from.


\chapter{Adding a title by Z39.50}
\label{\detokenize{copy-cataloging/adding-a-title-by-z39.50:adding-a-title-by-z39-50}}\label{\detokenize{copy-cataloging/adding-a-title-by-z39.50::doc}}
A. First search for a title using the \sphinxhref{../searching-for-a-title.md}{“Searching for a
title”} instructions.

\sphinxhref{../searching-for-a-title.md}{Click here for the “Searching for a title”
instructions}

B. If a title cannot be found, go to “Cataloging” and click on the “New
from Z39.50/SRU” button.

C. Fill out the form by searching for your title by ISBN, UPC, title, or
author - or a combination of any of these (to search by UPC, scan the
UPC code into the ISBN field) - then click on the “Search” button.

D. If you find a match, click on the “MARC” or the “Card” link to
pre-view the record and make sure that the record is a good match.
Generally speaking, the more data you see on the “Card” link, the better
the record.

E. Once you have chosen a record to add to our catalog, click on the
“Import” link to add the record to our catalog.

F. Once the record has been imported you can make changes to the MARC
record as necessary and then click the “Save” button to finalize the
import.

G. After the bibliographic record has been saved, you will be redirected
to the add items page so you can add your items to the new record.

\sphinxhref{https://github.com/will1410/cataloging-training/tree/09bcc3049af02f32a67b2f3dad708bbf3fd46050/.very-basic-cataloging/adding-an-item.md}{For more information about adding items, click
here}

H. If you don’t find any items via a Z39.50 it is not a bad idea to
repeat the search a couple of different ways.

You can re-do the search using different data.

You can select different catalogs to search in (not all catalogs are
searched by default).

I. If no search returns a result, you should move on to \sphinxhref{adding-a-title-from-shareit.md}{adding a title
through ShareIt.}

\sphinxhref{adding-a-title-from-shareit.md}{Click here for the “Adding a title from ShareIt”
instructions}


\chapter{Adding a title from ShareIt}
\label{\detokenize{copy-cataloging/adding-a-title-from-shareit:adding-a-title-from-shareit}}\label{\detokenize{copy-cataloging/adding-a-title-from-shareit::doc}}

\section{Search for and select the title/s to download}
\label{\detokenize{copy-cataloging/adding-a-title-from-shareit:search-for-and-select-the-title-s-to-download}}
A. Log in to ShareIt (if you don’t know how to log into ShareIt, please
call NEKLS).

B. Search for a title by ISBN/UPC/Title/Author as you would for any
other title.

C. Once you get a list of possible matches and find a record that
matches, click on a library name to open the title’s record.

D. Once the title record is open, click on the “Download record” button
to download the MARC record for this title.

E. This will download the records as a file onto your computer - usually
into your “Downloads” folder.


\section{To import the record/records into NExpress:}
\label{\detokenize{copy-cataloging/adding-a-title-from-shareit:to-import-the-record-records-into-nexpress}}\begin{enumerate}
\setcounter{enumi}{5}
\item {} 
Open the Tools menu in NExpress.

\end{enumerate}
\begin{enumerate}
\setcounter{enumi}{6}
\item {} 
Select “Stage marc records for import.”

\end{enumerate}

H. Click on “Browse” and go to the folder where the record was
downloaded - usually the “Downloads” folder on your computer.
\begin{enumerate}
\item {} 
Select the file to import and click on “Upload file.”

\end{enumerate}

J. Follow the hints on the dropdown menus and click on “Stage for
import.”

K. Once the items are staged, click on “Manage staged records” to finish
the import.
\begin{enumerate}
\setcounter{enumi}{11}
\item {} 
Click on “Import this batch into the catalog.”

\end{enumerate}

M. After the records have been imported, you can go to the new records
by clicking on the biblio numbers in the right hand column on the import
items table.

N. Once you’ve gone to the item’s record, you can add your items by
following the \sphinxhref{../title-already-exists/adding-an-item.md}{adding items
instructions.}

\sphinxhref{../title-already-exists/adding-an-item.md}{For more information about adding items, click
here}


\chapter{Original cataloging}
\label{\detokenize{original-cataloging/README:original-cataloging}}\label{\detokenize{original-cataloging/README::doc}}

\chapter{Adding a brief record}
\label{\detokenize{original-cataloging/adding-a-brief-record:adding-a-brief-record}}\label{\detokenize{original-cataloging/adding-a-brief-record::doc}}
If you can’t find a record already in NExpress, or a record via Z39.50
or a record in Shareit, you may need to create an original catalog
record.

Go to Cataloging (More\textendash{}\textgreater{}Cataloging), and choose New Record \textendash{}\textgreater{} Brief
Records Framework.

Fill in the following fields accordingly, if applicable:
\begin{itemize}
\item {} 
000 \textendash{} Leader - click in this field and let it autofill

\item {} 
008 \textendash{} Fixed length data elements - click in this field and let it
autofill

\item {} 
020 or 024

\item {} 
020 \textendash{} ISBN number if the item has one (books)

\item {} 
024 \textendash{} UPC code if the item doesn’t have an ISBN (DVDs, Blu-ray
discs, music CDs)

\item {} 
100 \textendash{} Author (if applicable)

\item {} 
245 \textendash{} Title (required)

\item {} 
250 \textendash{} edition (if applicable)

\item {} 
264 \textendash{} Publication information

\item {} 
a - place

\item {} 
b - publisher

\item {} 
c - year (most important piece of information here)

\item {} 
300 \textendash{} physical description. Please at least record the number of
pages, number of discs.

\item {} 
490 \textendash{} series statement (if applicable)

\item {} 
a - series name

\item {} 
v - volume number

\item {} 
500 \textendash{} add note if applicable

\end{itemize}

Once you save the record, you can add your items by following the
\sphinxhref{../title-already-exists/adding-an-item.md}{adding items
instructions.}

\sphinxhref{../title-already-exists/adding-an-item.md}{For more information about adding items, click
here}

If you are uncomfortable creating a new record, contact
\sphinxhref{mailto:nexpresshelp@nekls.org}{nexpresshelp@nekls.org}


\chapter{ILL cataloging}
\label{\detokenize{ill-cataloging/README:ill-cataloging}}\label{\detokenize{ill-cataloging/README::doc}}

\chapter{ILL cataloging stub records}
\label{\detokenize{ill-cataloging/ill-cataloging-stub-records:ill-cataloging-stub-records}}\label{\detokenize{ill-cataloging/ill-cataloging-stub-records::doc}}

\chapter{ILL cataloging brief records}
\label{\detokenize{ill-cataloging/ill-cataloging-brief-records:ill-cataloging-brief-records}}\label{\detokenize{ill-cataloging/ill-cataloging-brief-records::doc}}
Cataloging ILLs into Koha, Best Practices
\begin{enumerate}
\item {} 
Go to ​Cataloging​ (​More​\(\rightarrow\)​Cataloging​)

\item {} 
Click on the button, ​New record​ ​\(\rightarrow\)​ ​Fast Add for ILL and Temporary
Circulation​. The form for ​Add MARC​ record appears.

\item {} 
Click in the ​000​ and ​008​ fields and let them auto-fill

\item {} 
In the ​245a​ (Title) field, type in caps, “LIBRARY NAME ILL” Title
of material.

\item {} 
Click ​Save​.

\item {} 
The ​Add item​ form appears. Fill in the following: a. 8-Koha
collection​: ILL (by using this collection code, it remains hidden in
the OPAC) b. c-Shelving location​ (if appropriate) c. p-Barcode​ (can
be what’s on the material/a temporary one of your choosing) d.
​y-Koha item type​: already auto-set to Borrowed from another library

\item {} 
Click ​Add item​ to save the information

\end{enumerate}

Other ILL processes 8. Place a hold for the library patron getting the
item, and trigger the hold for pick up.
\begin{enumerate}
\item {} 
Place a second hold for a staff account, so when the item is returned
and possibly checked into Koha, the staff member checking it in,
knows to give it to the ILL person to check in on that system and
delete the info from Koha.

\item {} 
When the ILL item is returned, delete the item in Koha. No need to
delete the bib record, unless you want to \textendash{} a script deletes empty
records each week.

\end{enumerate}


\chapter{Indices and tables}
\label{\detokenize{index:indices-and-tables}}\begin{itemize}
\item {} 
\DUrole{xref,std,std-ref}{genindex}

\item {} 
\DUrole{xref,std,std-ref}{modindex}

\item {} 
\DUrole{xref,std,std-ref}{search}

\end{itemize}



\renewcommand{\indexname}{Index}
\printindex
\end{document}